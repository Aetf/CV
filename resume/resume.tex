\documentclass{tccv}
\usepackage[english]{babel}

\begin{document}

\part{Peifeng Yu}

\personal
    [unlimitedcodeworks.xyz]
    {2401 Lancashire Dr, Apt 1A\newline Ann Arbor, MI 48105}
    {+1 743 239 2157}
    {peifeng@umich.edu}

\section{Education}

\begin{yearlist}

\item{Present}
     {M.S. in Computer Science and \newline Engineering}
     {University of Michigan, USA}

\item{2015}
     {B.Eng. in Software Engineering}
     {Xi'an Jiaotong University, China}

\end{yearlist}


\section{Software skills}

\begin{factlist}

\item{Languages}
     {C++, Python, Java, C\#, \LaTeX, Shell, HTML/CSS/JS, Haskell}

\item{Tools}
     {git, Clang/LLVM, LLDB, GDB, MySQL, Subversion}

\item{Libraries}
     {Qt, KDE, numpy, matplotlib}
\end{factlist}

\section{Rewareds}

\begin{yearlist}
    \item{2014}
         {Google Excellent Scholarship}
         {\$1500, only one in XJTU}
    \item{2014}
         {Meritorious Winner}
         {Interdisciplinary Contest in Modeling}
    \item{2013}
         {Silver Medal of the ACM-ICPC}
         {Asia China Shaanxi Provincial Programming Contest}
    \item{2013}
         {Fuji Xerox (China) Scholarship}
         {RMB5000, 4 from top 20\%}
    \item{2012}
         {``Lu Shidi'' Scholarship}
         {RMB6000, 2 from top 10\%}
\end{yearlist}

\section{Project experience}

\begin{eventlist}

\item{May 2016 -- August 2016}
     {Google Summer of Code}
     {LLDB Support for KDevelop}

Software development for KDevelop (\href{https://www.kdevelop.org/}{kdevelop.org}) from KDE community:
LLDB debugger backend for KDevelop that enables debugging C/C++ programs in KDevelop using LLDB.
Architectural improvments to KDevelop C/C++ debugger to allow choosing between multiple debugger backends.

\vspace{0.2em}

Project page: \href{https://goo.gl/F6JsQd}{https://goo.gl/F6JsQd}

Final report post: \href{https://goo.gl/8uTdi3}{https://goo.gl/8uTdi3}

\item{September 2015 -- December 2015}
     {Advanced Compiler course project}
     {Software Low Power Mode}

Worked in a group of 4 people. Designed a runtime power saving optimization platform that can re-compile 
program to use more power efficient instructions at runtime, with minimal overhead. Implementation based on 
Protean code which is a runtime recompiling framework.

Protean code paper: 
\href{https://goo.gl/y3UBsP} {https://goo.gl/y3UBsP} 

\item{November 2013 -- June 2013}
     {Software Project Management course project}
     {Pixel Cube Game for LeapMotion}

Worked in a group of 8. 3D pixel painting using hand gestures, powered by LeapMotion and Microsoft WPF. 
Developed the in-game 3D menu that can be triggered by sliding fingures downwards.

Project page: \href{https://github.com/Aetf/PixelCube}{https://github.com/Aetf/PixelCube}

\item{July 2012 -- September 2013}
     {NSKeyLab (\href{http://nskeylab.xjtu.edu.cn/site/lab/}{http://nskeylab.xjtu.edu.cn/site/lab/})}
     {Security-Enhanced Android}

One year research program in the Lab. Focused on the implementation of multi-domain data separation. Ported 
the SELinunx policy compile tool chain to Android (4.2) platform. Implemented domain management and storage 
system service in both native user space and Android framework.

\item{May 2012 -- July 2012}
     {Microsoft Student Technology Club}
     {Mouse Control using Kinect}

Early exploration of UI opportunites offered by 3D sensor Kinect. Developed algorithm to smooth mouse 
movement by mapping hand movement into a cylinder surface. The algorithm can adapt to different body 
parameters.

Project page: \href{https://github.com/Aetf/KinectControl}{https://github.com/Aetf/KinectControl}

\item{2012 -- Present}
     {KDE, Mozilla and others}
     {Open Source Contribution}

Implemented recover instruction for multiple MIR instruction in JIT engine in Mozilla Firefox.

Fixed bugs in KDevelop GDB debugger backend.

Contributed to open source projects on GitHub: Mono, kmscon, CuteMarkEd, hid-apple-patched, 
DynamicTextures.

\end{eventlist}

\end{document}
