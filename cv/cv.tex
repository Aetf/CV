\documentclass[letterpaper,11pt]{article}
\usepackage{cv,booktabs,fontawesome}
\usepackage[utf8]{inputenc}
\usepackage[T1]{fontenc}
\usepackage{hyperref}

% Selectively show some content or not
\usepackage{comment}
%% non academia related thing goes into extra
\newenvironment{extra}{}{}
%% non academia related thing goes into extra
\newenvironment{excurr}{}{}
%% toggles (comment out all the following to create full version)
%\excludecomment{extra}
%\excludecomment{excurr}

% Change color to blue
\def\headcolor{\color[rgb]{0,0,0.5}}

% Space before section headings
\titlespacing{\section}{0pt}{3ex}{1ex}

% bib reference font
\usepackage[sfdefault,lf,t]{carlito}
\renewcommand{\bibfont}{\normalfont\fontsize{10}{12.4}\sffamily}
\usepackage{inconsolata}

\name{Peifeng Yu}
\info{%
    \faicon{home}       & \href{https://unlimited-code.works}{unlimited-code.works} \\
    \faicon{github}     & \href{https://github.com/Aetf}{Aetf} \\
    \faicon{phone}      & +1 734 239 2157 \\
    \faicon{envelope}   & peifeng@umich.edu
}

\bibliography{cv}

%\addtocategory{books}{MWH3,ITSM91,ITSM94,expsmooth08}
\addtocategory{workshop}{hotos17}
\addtocategory{poster}{yu18michai}
\addtocategory{poster}{sysml18}
\addtocategory{papers}{fluid:mlsys21}
\addtocategory{papers}{salus:mlsys20}
\addtocategory{papers}{fb:ieeemicro}

\begin{document}
\maketitle

\section{Education and Qualifications}

\begin{tabular}{llll}
    Present        & PhD  & Computer Science and Engineering & University of Michigan \\
    2017           & MS   & Computer Science and Engineering & University of Michigan \\
    2015           & B.Eng. & Software Engineering & Xi'an Jiaotong University
\end{tabular}

\begin{publications}
    %\printbib{phd}
    %\printbib{books}
    \printbib{papers}
    \printbib{workshop}
    %\printbib{poster}
    \printbib{subpapers}
    %\printbib{conferences}
    %\printbib{bookreviews}
    %\printbib{editorials}
\end{publications}

\section{Work Experience}
\begin{itemize}
    \item Internship at Facebook \hfill \textit{From May., 2019 to Aug., 2019}
    \begin{itemize}
        \item Build fleet-wide GPU utilization regression detection and attribution dashboard
        \item Discover and fix data consistency issue in GPU performance data.
        \item Identify optimization opportunities and give improvement suggestions via automated 
emails.
    \end{itemize}

\end{itemize}

\section{Research Experience}
\begin{itemize}
    \item Fluid: Resource-aware Hyperparameter Tuning Engine
    Engine \hfill \textit{From May, 2020 to Jan., 2021}
    \begin{itemize}
        \item Generalized Hyperparameter tuning execution engine using efficient
        heuristics with theoretical guarantees to solve the scheduling problem.
        \item Improve resource utilization by both inter- and intra-GPU training
        mechanisms
        \item Code open sourced at \url{https://github.com/SymbioticLab/fluid}
    \end{itemize}
    \item Salus: Fine-Grained GPU Sharing for Deep Learning Applications \hfill \textit{From Sep., 2016 to Apr., 2020}
    \begin{itemize}
        \item Fine-grained GPU sharing by providing missing primitives: fast 
switching and memory sharing.
        \item Improves GPU utilization for hyper-parameter tuning by $2.38\times$, and for DL 
inference applications by $42\times$ over not sharing the GPU.
        \item Code open sourced at \url{https://github.com/SymbioticLab/Salus}
    \end{itemize}

    \item Deep Tree: SQL Injection Detection by the Power of Deep Learning \hfill \textit{From Sep., 2016 to Mar., 2017}
    \begin{itemize}
        \item Tree-based CNN for SQL statements classification with 94.7\% accuracy for injection 
detection.
        \item Compiled new SQL statements dataset of 4161 samples.
        \item Code open sourced at \url{https://github.com/Aetf/tensorflow-tbcnn}
    \end{itemize}

    \item System Design of Streaming Video Analysis Application in Storm \hfill \textit{From Apr., 2016 to July, 2016}
    \begin{itemize}
        \item Storm topology with several video classification, captioning and object tracking 
workloads.
        \item Latency and throughput analysis on 3 GPU servers, to understand the relationship 
between parallelism hint and performance.
        \item Summer research in Clarity Lab.
    \end{itemize}

\begin{extra}
    \item Evaluation of Graphical Keyboard User Interface \hfill \textit{From Sep., 2015 to Dec., 2015}
    \begin{itemize}
        \item Evaluated of two GKUI applications completing different tasks. Operating systems were also
included as a variable in the experiment.
        \item The result indicates significant improvements using GKUI in both tasks.
        \item Course research project for Introduction to HCI Research at University of Michigan.
    \end{itemize}
\end{extra}

    \item Neural Network Classifier with Generalized Correntropy Loss \hfill \textit{From Jan., 2015 to
May, 2015}
    \begin{itemize}
        \item Bachelor's thesis.
        \item Implemented a neural network classifier using generalized correntropy loss function.
        \item Analyzed the classifier's behavior under varied order parameters in generalized correntropy loss
function.
    \end{itemize}

\begin{extra}
    \item Application data isolation using SEAndroid in NSKeyLab \hfill \textit{From Nov., 2013 to Nov., 2014}
    \begin{itemize}
        \item Focused on the implementation of multi-domain data isolation.
        \item Ported the SELinux policy compile tool chain to Android.
        \item Implemented domain management and storage system service in both native user space and Android
framework.
    \end{itemize}
\end{extra}

\begin{extra}
    \item Summer Practice in NSKeyLab (the Ministry of Education Key Lab for Intelligent Networks and \\ Network
    Security on Network Traffic Capture and Analysis) \hfill \textit{From July, 2012 to Sep, 2013}
    \begin{itemize}
        \item Mainly engaged in the development of network traffic capture and reconstruction algorithm.
        \item Used techniques include WinPcap, WPF, TCP stream reassembly and HTTP reconstruction.
        \item Analyzed 5 high speed download traffic samples and tens of HTTP traffic samples.
    \end{itemize}
\end{extra}

\begin{extra}
    \item Graphical Data Quality Management System based on IP-MAP \hfill \textit{From Nov., 2012 to May,
2013}
    \begin{itemize}
        \item Professional graphical software offering a specialized platform for data quality management based on IP-MAP.
        \item Funded by the national innovation fund project of College Students of Xi’an Jiaotong University.
        \item Third prize by Xian Jiaotong University in the “Tengfei Cup” undergraduate
curricular academic science and technology competition.
        \item Participated as team leader and programmer.
    \end{itemize}
\end{extra}

\end{itemize}


\section{Project Experience}

\begin{itemize}
    \item Activities on GitHub \hfill \textit{From 2012 to present}
    \begin{itemize}
        \item Contributed to several projects: KeepassXC, kmscon, Mono, CuteMarkEd, hid-apple-patched.
        \item Side projects including a bencode library, network traffic capture and analyze, torrent list migration between uTorrent and qBittorrent.
    \end{itemize}

    \item LLDB Support for KDevelop (Google Summer of Code) \hfill \textit{From May, 2016 to Aug., 2016}
    \begin{itemize}
        \item Extended KDevelop C/C++ debugger architecture to allow multiple debugger backends.
        \item Added LLDB backend for KDevelop, enabling C/C++ debugging through LLDB.
        \item Successful Google Summer of Code project. Link:
\url{https://summerofcode.withgoogle.com/archive/2016/projects/6014826014834688/}
        \item Keep contributing to the KDevelop after the GSoC period.
    \end{itemize}

    \item Contribution to open source Mozilla project \hfill \textit{From June, 2014 to 2015}
    \begin{itemize}
        \item Fixed several bugs (features) in JavaScript JIT Engine.
        \item Implemented recover instruction for multiple MIR instructions.
        \item Got Mozillian membership.
    \end{itemize}

    \item Software Low Power Mode Based on Protean Code \hfill \textit{From Sep., 2015 to Dec., 2015}
    \begin{itemize}
        \item A runtime power saving optimization platform based on Protean Code triggered by OS power events.
        \item Implemented devectorization pass which disables SLP instructions when batteries are running out.
        \item Course project for Advanced Compiler in University of Michigan, worked in a group of 4.
    \end{itemize}

    \item Pixel Cube Game for LeapMotion \hfill \textit{From Nov., 2013 to June, 2014}
    \begin{itemize}
        \item 3D pixel painting using hand gestures, powered by LeapMotion
        \item Course project for Software Project Management in Xi'an Jiaotong University, worked in a group
of 8.
    \end{itemize}

    \item Mouse Control with Kinect \hfill \textit{From May, 2012 to July, 2012}
    \begin{itemize}
        \item Developed algorithm to smooth mouse movement by projecting hand movement onto a 
cylindrical surface.
        \item The algorithm can adapt to different body parameters.
    \end{itemize}

\end{itemize}


\section{Awards and Scholarships}

\begin{tabular}{lp{15cm}}
    2014 & Google Excellent Scholarship (\$1500, only one in Xi'an Jiaotong University) \\
    2014 & Meritorious Winner of Interdisciplinary Contest in Modeling \\
    2013 & Silver Medal of the ACM-ICPC Asia China
            \newline Shaanxi Provincial Programming Contest \\
    2013 & Fuji Xerox (China) Scholarship (RMB5000, 4 of top 20\% student in the major) \\
    2013 & Merit Student in Xi'an Jiaotong University \\
    2012 & First prize for MCM/ICM of Xi'an Jiaotong University \\
    2012 & Third prize for ACM Programming Contest of Xi'an Jiaotong University \\
    2012 & Third prize for ``Tengfei Cup'' Undergraduate
            \newline Curricular Academic Science and Technology Competition \\
    2011 & Excellent Student Cadre in Xi'an Jiaotong University  \\
    2011 & ``Lu Shidi'' Scolarship (RMB6000, 2 of top 10\% students in the major)
\end{tabular}

\end{document}
